% ----------------------------------------------------------------------
%  Pracovní úkoly
% ----------------------------------------------------------------------
\section{Pracovní úkoly}

\begin{enumerate}
\item Určete závislost povrchového napětí \(\sigma\) na objemové koncentraci \(c\) roztoku etylalkoholu ve vodě odtrhávací metodou.

\item Sestrojte graf této závislosti.

\end{enumerate}

% ----------------------------------------------------------------------
%  Teoretická část
% ----------------------------------------------------------------------
\section{Teoretická část}

Povrchovým napětím \(\sigma\) nazýváme kolmou sílu působící na jednotkovou délku na povrch určité látky. Zároveň je tato síla stejná ve všech místech povrchu. Povrchovým napětím se zabýváme zejména u kapalin. Velikost napětí je závislá na teplotě a čistotě kapalin. Látky, které ovlivňují tuto hodnotu, se nazývají povrchově aktivní.

Závislost povrchového napětí na koncentraci budeme měřit tzv. odtrhávací metodou.

% ----------------------------------------------------------------------
%  Výsledky a zpracování měření
% ----------------------------------------------------------------------
\section{Výsledky a zpracování měření}

\subsection{Laboratorní podmínky}

    Měření bylo prováděno za laboratorních podmínek uvedených v tabulce \ref{tab:lab_pod}.

    \begin{table}[h]
        \centering
        \begin{tabular}{|c|c|c|} 
        \hline
            t / °C & p / hPa & vlhkost / \%RH  \\ 
        \hline
            24,0(4)   & 989,2(20)   & 42,0(25)            \\
        \hline
        \end{tabular}
        \caption{Laboratorní podmínky}
        \label{tab:lab_pod}
    \end{table}

\subsection{Podnadpis}

    
% ----------------------------------------------------------------------
%  Diskuse výsledků
% ----------------------------------------------------------------------			
\section{Diskuse výsledků}

% ----------------------------------------------------------------------
%  Závěr
% ----------------------------------------------------------------------
\section{Závěr}
