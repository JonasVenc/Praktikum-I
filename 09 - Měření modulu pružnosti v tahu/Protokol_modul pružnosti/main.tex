% ----------------------------------------------------------------------
%  Základní nastavení dokumentu - musí být na začátku každého tex
%  (pořadí příkazů v této části je důležité!)
% ----------------------------------------------------------------------

%  Typ dokumentu - článek, prezentace aj.
% Standardní základ
%\documentclass[a4paper,10pt]{article} 
%\usepackage[letterpaper]{geometry}
%\geometry{verbose,tmargin=1.5cm,bmargin=2cm,lmargin=2cm,rmargin=2cm}

% Alternativní verze - vhodnější formát papíru (větší hustota textu)
%\documentclass[10pt]{scrartcl}
%\KOMAoptions{DIV=20} % formát papíru a odsazení od jeho okrajů
\documentclass{prepareprotokol} %všechny balíčky pro protokol + titulní strana

%  Kódování výstupu - aby šlo z pdf kopírovat včetně háčků a čárek
\usepackage[T1]{fontenc} 

%  Kódování vstupu - v kódu lze použít háčky a čárky
\usepackage[utf8]{inputenc} 

%  Základní typografická pravidla češtiny/slovenštiny
\usepackage[czech]{babel} 
%\usepackage[slovak]{babel}
% použijte jen jeden z příkazů


%  Lépe vypadající písmo pro T1 kódování
\usepackage{lmodern} 
% je možno zakomentovat, nepoužívá-li se T1 kódování




%  Formátování stránek, empty = odstraní číslování
% \pagestyle{empty}

%  Řádkování
\linespread{1.1}


% ----------------------------------------------------------------------
%  Doplňující balíčky
% ----------------------------------------------------------------------

%  Po desetinné čárce v matematickém módu se nevytvoří mezera
\usepackage{icomma} 

%  Umožňuje pracovat s grafikou
\usepackage{graphicx}

%  Umožňuje použít dva obrázky vedle sebe
\usepackage{subcaption}

%  Pro vkládání obrázků ve formátu eps (např. z gnuplot)
\usepackage{epstopdf} 

%  Automaticky odsadí i první paragraf v každé sekci
\usepackage{indentfirst}

%  Umožňuje rozdělovat obsah na více sloupců
\usepackage{multicol}
\usepackage{booktabs}
\usepackage{pgffor}

%  Umožňuje používat hypertextové odkazy, nastavuje jejich vlastnosti
\usepackage[unicode]{hyperref}


% ----------------------------------------------------------------------
%  Matematika
% ----------------------------------------------------------------------

%  Lepší zobrazování matematiky (rozšíření sum o \limits atd.)
\everymath{\displaystyle}

%  Široké spektrum příkazů pro matematiku
% (Umožní např. psát přes \mathbb{N/R/Q/..} množiny čísel)
\usepackage{amsmath,amssymb}

%  Velikost fontu matematických výrazů v dokumentu lze pro danou
% základního fontu dokumentu upravit pomocí:
% \DeclareMathSizes{X}{Y}{Z}{U} kde:
% X je velikost fontu v dokumentu, pro kterou se matematika upraví
% Y je standartní velikost fontu matematiky
% Z je velikost fontu zmenšených (vnořených výrazů)
% U je velikost fontu ještě více zmenšených (vnořených výrazů).
\DeclareMathSizes{10}{10}{8}{7}

%  Široké spektrum příkazů pro fyziku
\usepackage{physics}

%  Psaní SI jednotek
\usepackage{siunitx}

%  Nám bližší zápis písmene epsilon
\AtBeginDocument{%
%\let\phi\varphi
\let\epsilon\varepsilon
}


% ----------------------------------------------------------------------
%  Pro češtinu/slovenštinu
% ----------------------------------------------------------------------

%  Lokalizace některých názvů do češtiny/slovenštiny
\addto\captionsczech{\renewcommand{\figurename}{Obr.}}
\addto\captionsczech{\renewcommand{\tablename}{Tab.}}
%\addto\captionsczech{\renewcommand{\refname}{Reference}}

\addto\captionsslovak{\renewcommand{\figurename}{Obr.}}
\addto\captionsslovak{\renewcommand{\tablename}{Tab.}}
%\addto\captionsslovak{\renewcommand{\refname}{Reference}}

% Odkomentujte následující příkaz, máte-li stažený balíček encxvlna 
% (nutno stáhnout manuálně)
%\usepackage{encxvlna} %vloží nezlomitelné mezery k jednopísmenným



% ----------------------------------------------------------------------
%  Soubor s makry
% ----------------------------------------------------------------------
%% ----------------------------------------------------------------------
%  Identifikace protokolu (příkazy lze použít v celém dokumentu)
% ----------------------------------------------------------------------

%  Nastaví autora, název, datum, skupinu měření apod. 
\newcommand{\Institute}{FJFI~ČVUT~v~Praze}
%\newcommand{\Subject}{Základy fyzikálních měření}
\newcommand{\Subject}{Fyzikální praktikum II}  %odkomentujte dle potřeby

%  Máte-li více spoluměřících než jednoho, vložte jen jejich příjmení
\newcommand{\Author}{Jméno autora}
\newcommand{\Coauthor}{Jméno kolegy} 
\newcommand{\Group}{Pátek} %den, kdy chodíte na praktika, nikoli obor
\newcommand{\Circle}{2} %číslo skupiny v rámci praktika, nikoli kruh

%  Tato část bude v každém protokolu jiná, nezapomeňte upravit!
\newcommand{\Title}{Úloha 0 -- Psaní vzorového protokolu}
\newcommand{\Labdate}{1.1.2017} %datum měření, nikoli datum odevzdání
\newcommand{\Worktime}{5 h} %jak dlouho vám trvalo vypracování protokolu




% ----------------------------------------------------------------------
%  Vlastní příkazy
% ----------------------------------------------------------------------


%  Matematika
\newcommand{\ee}{\mathrm{e}} %eulerovo číslo
\newcommand{\ii}{\mathrm{i}} %imaginární jednotka

% Jednotky
%\newcommand{\unit}[1]{\,\mathrm{#1}} %jednotky zadávejte pomocí tohoto příkazu
\renewcommand{\deg}{\ensuremath{\mathring{\;}}} %symbol stupně
\newcommand{\celsius}{\ensuremath{\deg\mathrm{C}}} %stupně celsia

%(hodnota plus mínus chyba) jednotka
\newcommand{\hodn}[3]{(#1 \pm #2)\unit{#3}} 

%veličina [jednotka] do hlavičky tabulky
\newcommand{\tabh}[2]{\ensuremath{#1\,[\mathrm{#2}]}} 


% ----------------------------------------------------------------------
%  Vlastní příkazy
% ----------------------------------------------------------------------


%  Matematika
\newcommand{\ee}{\mathrm{e}} %eulerovo číslo
\newcommand{\ii}{\mathrm{i}} %imaginární jednotka

% Jednotky
%\newcommand{\unit}[1]{\,\mathrm{#1}} %jednotky zadávejte pomocí tohoto příkazu
\renewcommand{\deg}{\ensuremath{\mathring{\;}}} %symbol stupně
\newcommand{\celsius}{\ensuremath{\deg\mathrm{C}}} %stupně celsia

%(hodnota plus mínus chyba) jednotka
\newcommand{\hodn}[3]{(#1 \pm #2)\unit{#3}} 

%veličina [jednotka] do hlavičky tabulky
\newcommand{\tabh}[2]{\ensuremath{#1\,[\mathrm{#2}]}}

%  nachází se ve složce /tex/



% ----------------------------------------------------------------------
%  Nastavení odkazů a výsledného pdf
% ----------------------------------------------------------------------
\hypersetup{
colorlinks=true, 
citecolor=blue, 
filecolor=blue, 
linkcolor=blue,
urlcolor=blue, 
%pdftitle={\title},    % title
pdfauthor={Jonáš Venc},     % author
pdfsubject={Protokol},   % subject of the document
pdfcreator={Jonáš Venc},   % creator of the document
%     pdfproducer={Producer}, % producer of the document
%     pdfkeywords={keywords}, % list of keywords
pdfnewwindow=true,      % links in new window
}

% ----------------------------------------------------------------------
%  Začátek dokumentu - formátování na výstup
% ----------------------------------------------------------------------

\praktikum{I} %číslo praktika (I, II nebo III)
\autor{Jonáš Venc} %jméno autora
\datum{15.\,3.\,2024} %datum měření
\cislo{9} %číslo úlohy
\nazev{Měření modulu pružnosti v tahu} %název úlohy

\begin{document}

\maketitle

% ----------------------------------------------------------------------
%  Tělo dokumentu
% ----------------------------------------------------------------------

\setlength{\parindent}{0.5cm}

% ----------------------------------------------------------------------
%  Protokol

\pagenumbering{arabic}  % číslování stránek čísly
% ----------------------------------------------------------------------
%  Pracovní úkoly
% ----------------------------------------------------------------------
\section{Pracovní úkoly}

\begin{enumerate}
\item Změřte teplotní závislost povrchového napětí destilované vody $\sigma$ v rozsahu teplot od 23°C do 70°C metodou bublin.

\item Měřenou závislost znázorněte graficky, do grafu vyneste chybové úsečky a tabulkové hodnoty. Závislost aproximujte kvadratickou funkcí.

\end{enumerate}

% ----------------------------------------------------------------------
%  Teoretická část
% ----------------------------------------------------------------------
\section{Teoretická část}

Na obrázku \ref{fig:aparatura-povrchove-napeti} je znázorněna aparatura pro měření závislosti povrchového napětí na teplotě podle [1].

\begin{figure}[h]
    \centering
    \includegraphics[width=0.75\linewidth]{14 - Studium teplotní závislosti povrchového napětí//Protokol - studium povrchového napětí//img/Aparatura.png}
    \caption{Aparatura pro měření povrchového napětí}
    \label{fig:aparatura-povrchove-napeti}
\end{figure}

Měřená kapalina se nachází v nádobce N. Do této nádoby je zavedena kapilára K, která je spojena s vnějším prostorem. Je ponořena těsně pod hladinu a my tak můžeme zanedbat hydrostatický tlak kapaliny působící na konec této trubice. V aspirátoru A je umístěna voda jejíž odtok lze regulovat přítlačnou svorkou PS. Prostor nad kapalinou je propojen se uzavřenou nádobou N a mikromanometrem M, který umožňuje měřit rozdíl tlaku. Aspirátor funguje na principu vodní vývěvy a tím snižuje tlak oproti tlaku ve vnějším prostoru. Díky tomuto rozdílu tlaků se u kapiláry začne vytvářet bublinka, až se nakonec oddělí a zvýší zpět tlak v prostoru nad hladinou.

Na vytlačovaná vzduch z kapiláry tedy působí hydrostatický tlak, který můžeme zanedbat a povrchové napětí napětí kapaliny $\sigma$. Uvnitř kulové plochy o poloměru $r$ se vytváří kapilární přetlak $\Delta p_\sigma$

\begin{equation}
    \Delta p_\sigma = \frac{2\sigma}{r}
\end{equation}

Tento přetlak je největší ve chvíli, kdy je poloměr $r$ bubliny nejmenší, tedy shodný s poloměrem kapiláry.

Pro určení tlakového rozdílu použijeme vztah

\begin{equation}
    \Delta p = d \rho g \sin{\alpha}
\end{equation}

kde $d$ je výška vodního sloupce v mikromanometru, $g$ tíhové zrychlení, $\rho$ je hustota kapaliny v mikromanometru a $\sin{\alpha}$ úhel sklonu mimkomanometrické trubice.

V našem případě je skol trubice 30° a můžeme tedy dosadit

\begin{equation}
    \sin{30^{\circ}} = \frac{1}{2}
\end{equation}

Uvažujeme maximální povrchové napětí a společně s (1) dostáváme

\begin{equation}
    \sigma = \frac{1}{4} d_{max} \rho g r
\end{equation}

kde $d_{max}$ je maximální výška hladiny v mikromanometru a $r$ poloměr trubice.

% ----------------------------------------------------------------------
%  Výsledky a zpracování měření
% ----------------------------------------------------------------------
\section{Výsledky a zpracování měření}

\subsection{Laboratorní podmínky}

    Měření bylo prováděno za laboratorních podmínek uvedených v tabulce \ref{tab:lab_pod}.

    \begin{table}[h]
        \centering
        \begin{tabular}{|c|c|c|} 
        \hline
            t / °C & p / hPa & vlhkost / \%RH  \\ 
        \hline
            24,3(4)   & 983,2(20)   & 38,9(25)            \\
        \hline
        \end{tabular}
        \caption{Laboratorní podmínky}
        \label{tab:lab_pod}
    \end{table}

\subsection{Měření povrchového napětí}

Do nádoby N je vložen teploměr pomocí kterého budeme odečítat teploty měřené kapaliny. Uvažujeme tedy maximální kapilární přetlak, proto odečítáme maximální výšku $d$ v mikromanometru za dané teploty. Na začátku, kdy má měřená destilovaná voda laboratorní teplotu, změříme povrchové napětí třikrát pro určení nejistoty typu A. Tato měření jsou znázorněna a vyhodnocena v tabulce \ref{tab:nejistota-A}.

\begin{table}[h]
\centering
\begin{tabular}{|c|cc|}
\hline
Číslo měření        & \multicolumn{1}{c|}{d / mm} & \sigma / 10^{-3} N/m \\ \hline
1                   & \multicolumn{1}{c|}{109}    & 69,31              \\ \hline
2                   & \multicolumn{1}{c|}{108}    & 68,68              \\ \hline
3                   & \multicolumn{1}{c|}{109}    & 69,31              \\ \hline
Aritmetický průměr  &                             & 69,10              \\ \hline
Směrodatná odchylka &                             & 0,04               \\ \hline
\end{tabular}
\caption{Měření nejistoty typu A}
\label{tab:nejistota-A}
\end{table}

kde směrodatná odchylka je rovna nejistotě typu A, kterou označíme $\psi_A$.

Poté jsme zapnuli zahřívání a průběžně zaznamenávali výšku $d$ při stoupající teplotě. Rovnoměrné zahřívání zajišťovalo magnetické míchátko. V tabulce \ref{tab:napeti-na-teplote} jsou zaznamenaná jednotlivá měření.

\begin{table}[h]
\centering
\begin{tabular}{|c|c|c|c|c|}
\hline
Teplota /  °C & d / mm & \sigma / 10^{-3} N/m & \psi_{\sigma} / 10^{-3} N/m & \psi / 10^{-3} N/m \\ \hline
24            & 109    & 69,31            & 2,74        & 2,74 \\ \hline
25            & 106    & 67,40            & 2,67        & 2,67 \\ \hline
27            & 105    & 66,77            & 2,65        & 2,65 \\ \hline
30            & 103    & 65,50            & 2,60        & 2,60 \\ \hline
33            & 101    & 64,23            & 2,55        & 2,55 \\ \hline
37            & 100    & 63,59            & 2,53        & 2,53 \\ \hline
40            & 99     & 62,95            & 2,50        & 2,50 \\ \hline
45            & 98     & 62,32            & 2,48        & 2,48 \\ \hline
48            & 97     & 61,68            & 2,46        & 2,46 \\ \hline
50,5          & 96     & 61,05            & 2,43        & 2,43 \\ \hline
55            & 95     & 60,41            & 2,41        & 2,41 \\ \hline
58            & 94     & 59,77            & 2,39        & 2,39 \\ \hline
60,5          & 94     & 59,77            & 2,39        & 2,39 \\ \hline
65            & 93     & 59,14            & 2,36        & 2,36 \\ \hline
68            & 92     & 58,50            & 2,34        & 2,34 \\ \hline
70            & 92     & 58,50            & 2,34        & 2,34 \\ \hline
\end{tabular}
\caption{Měření závislosti povrchového napětí na teplotě}
\label{tab:napeti-na-teplote}
\end{table}

kde $\psi_{\sigma}$ je nejistota typu B jednotlivých měření. Je určena pomocí zákonu přenosu chyb

\begin{equation}
    \sigma^2 = \sum_{i=1}^{N} \left( \frac{\partial f}{\partial x_i} \right)^2 \sigma^2_{x_i}
\end{equation}

\newpage

odtud dostáváme

\begin{equation}
    \psi_{\sigma} = \sigma \sqrt{\left( \frac{\psi_d}{d} \right)^2 + \left( \frac{\psi_r}{r} \right)^2}
\end{equation}

kde $\psi_d$ je chyba výšky vodního sloupce a $\psi_r$ je chyba poloměru kapiláry. Chyba $\psi_r$ je zadána ve studijním textu jako $\psi_r$ = 0,01 mm a chyba $\psi_d$ byla určena jako jeden dílek na stupnici, tedy $\psi_d$ = 1 mm. Tato hodnota byla odhadnuta jako polovina nejmenšího dílku a navíc nadhodnocena, protože trubice je nakloněna a meniskus kapaliny také může ztížit čitelnost.

Pro přehlednost můžeme dosadit a nejistotu $\psi_\sigma$ vyjádřit jako

\begin{equation}
    \psi_\sigma = \sigma \sqrt{\frac{1}{d^2} + \left( \frac{0,01}{0,26} \right)^2}
\end{equation}

kde za $d$ dosazujeme v milimetrech.

Celková nejistota $\psi$ je spočtena jako

\begin{equation}
    \psi = \sqrt{\psi^2_A + \psi^2_{\sigma}}
\end{equation}

Tuto závislost můžeme zanést do grafu



% ----------------------------------------------------------------------
%  Diskuse výsledků
% ----------------------------------------------------------------------			
\section{Diskuse výsledků}

% ----------------------------------------------------------------------
%  Závěr
% ----------------------------------------------------------------------
\section{Závěr}

% ----------------------------------------------------------------------

% ----------------------------------------------------------------------
%  Literatura

\section{Použitá literatura}		
\begingroup
\renewcommand{\section}[2]{}
% ----------------------------------------------------------------------
%  Reference
% ----------------------------------------------------------------------

% sem doplňujte použité zdroje
\begin{thebibliography}{9}
\bibitem{bib:zadani} Kolektiv KF. \emph{Návod: \Title} [Online]. [cit. \today]. \newline \url{http://praktikum.fjfi.cvut.cz/pluginfile.php/415/mod\_resource/content/test.pdf}

\bibitem{bib:chyby} Kolektiv KF. \emph{Chyby měření} [Online]. [cit. \today]. \newline \url{http://praktikum.fjfi.cvut.cz/documents/chybynav/chyby-o.pdf}

\end{thebibliography}

\endgroup
% ----------------------------------------------------------------------


% ----------------------------------------------------------------------
%  Příloha

% ----------------------------------------------------------------------

%\clearpage
				
%\clearpage


\end{document}

% ----------------------------------------------------------------------
%  Konec dokumentu
% ----------------------------------------------------------------------
