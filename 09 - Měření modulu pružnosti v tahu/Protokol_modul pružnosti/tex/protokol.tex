% ----------------------------------------------------------------------
%  Pracovní úkoly
% ----------------------------------------------------------------------
\section{Pracovní úkoly}

\begin{enumerate}
\item Změřte modul pružnosti v tahu E oceli z protažení drátu.

\item Změřte modul pružnosti v tahu E oceli a mosazi z průhybu trámku.

\item Výsledky měření graficky znázorněte, modul pružnosti určete pomocí lineární regrese.

\end{enumerate}

% ----------------------------------------------------------------------
%  Teoretická část
% ----------------------------------------------------------------------
\section{Teoretická část}

\subsection{Protažení drátu}

Pro měření modulu pružnosti \(E\) využijeme Hookův zákon

\begin{equation}
    \sigma = E \cdot \epsilon
\end{equation}

kde \(\sigma\) je normálové napětí a \(\epsilon\) relativní prodloužení. Tento zákon platí pouze v oblasti pružné deformace, proto budeme měření provádět i zpětně, abychom ověřili, že je deformace vratná a my jsme tak nepřekročili mez pružnosti.

\subsection{Průhyb trámku}

% ----------------------------------------------------------------------
%  Výsledky a zpracování měření
% ----------------------------------------------------------------------
\section{Výsledky a zpracování měření}

\subsection{Laboratorní podmínky}

    Měření bylo prováděno za laboratorních podmínek uvedených v tabulce \ref{tab:lab_pod}. Pro naše měření je ale důležité, aby hodnoty (jako například teplota vody procházející  trubicí) byly co možná nejvíce podobné po celou dobu.

    \begin{table}[h]
        \centering
        \caption{Laboratorní podmínky}
        \label{tab:lab_pod}
        \begin{tabular}{|c|c|c|} 
        \hline
            t / °C & p / hPa & vlhkost / \%RH  \\ 
        \hline
            22,9(4)   & 977,2(20)   & 34,7(25)            \\
        \hline
        \end{tabular}
    \end{table}

\subsection{Podnadpis}

    
% ----------------------------------------------------------------------
%  Diskuse výsledků
% ----------------------------------------------------------------------			
\section{Diskuse výsledků}

% ----------------------------------------------------------------------
%  Závěr
% ----------------------------------------------------------------------
\section{Závěr}
