% ----------------------------------------------------------------------
%  Pracovní úkoly
% ----------------------------------------------------------------------
\section{Pracovní úkoly}

\begin{enumerate}
\item Při zapnutém kompresoru proměřte časovou závislost teplot v obou rezervoárech. Současně zaznamenávejte elektrický příkon kompresoru a časovou závislost tlaků v aparatuře. Graficky znázorněte.

\item Vyhodnoťte chladicí a topný faktor zdroje tepla, graficky znázorněte závislosti ε = ε(ΔT) a τ = τ(ΔT)

\item Při vypnutém kompresoru proměřte časovou závislost teploty vody v obou rezervoárech. Graficky znázorněte a závislost fitujte exponenciálou. Diskutujte koeficienty naměřené závislosti.

\item Ze změny teploty rezervoáru po vypnutí kompresoru určete dolní odhad tepelných ztrát.

\end{enumerate}

% ----------------------------------------------------------------------
%  Teoretická část
% ----------------------------------------------------------------------
\section{Teoretická část}

Podstatou fungování tepelného čerpadla je předávání tepla z jednoho rezervoáru do druhého. Používají se různá média pro přenos tepla, v našem případě je v rezervoárech umístěna destilovaná voda, kde rezervoár, který se během fungování otepluje je označen červeně a druhý, který se ochlazuje má modrou barvu.

Tepelné čerpadlo je tvořeno uzavřeným okruhem, kde koluje pracovní látka 1,1,1,2-
tetrafluorethan. V oblasti modrého kolektoru se v měděném potrubí chladivo nachází v plynné fázi. Odtud je přečerpáváno do kompresoru, kde je zvyšován tlak a tím dochází ke zkapalnění. Zde začíná vysokotlaká část čerpadla. Během tohoto procesu látka odevzdává část energie do červeného kolektoru, což způsobuje zvyšování teploty vody uvnitř kolektoru. Odtud putuje chladivo v kapalném stavu přes vysokotlaký manometr do kolektoru. Zde se látka hromadí a je regulovaně propouštěna do expanzního ventilu. Zde je snižován tlak a kapalina se tím vypařuje. Začíná tak nízkotlaká část čerpadla. Tím odebírá teplo z okolí a snižuje teplotu modrého rezervoáru. Je zde také umístěn nízkotlaký barometr. Následuje teplotní čidlo, které reguluje expanzní ventil proto, aby se do kompresoru dostávala pouze plynná fáze. Za čidlem se nachází již kompresor, který takto celý okruh uzavírá, jak je znázorněno na obrázku.



% ----------------------------------------------------------------------
%  Výsledky a zpracování měření
% ----------------------------------------------------------------------
\section{Výsledky a zpracování měření}

\subsection{Laboratorní podmínky}

    Měření bylo prováděno za laboratorních podmínek uvedených v tabulce \ref{tab:lab_pod}. Pro naše měření je ale důležité, aby hodnoty (jako například teplota vody procházející  trubicí) byly co možná nejvíce podobné po celou dobu.

    \begin{table}[h]
        \centering
        \caption{Laboratorní podmínky}
        \label{tab:lab_pod}
        \begin{tabular}{|c|c|c|} 
        \hline
            t / °C & p / hPa & vlhkost / \%RH  \\ 
        \hline
            22,9(40)   & 977,2(20)   & 34,7(25)            \\
        \hline
        \end{tabular}
    \end{table}

\subsection{Podnadpis}

    
% ----------------------------------------------------------------------
%  Diskuse výsledků
% ----------------------------------------------------------------------			
\section{Diskuse výsledků}

% ----------------------------------------------------------------------
%  Závěr
% ----------------------------------------------------------------------
\section{Závěr}
