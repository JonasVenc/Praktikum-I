% ----------------------------------------------------------------------
%  Pracovní úkoly
% ----------------------------------------------------------------------
\section{Pracovní úkoly}

\begin{enumerate}
\item Okalibrujte pomocí bodu tání ledu, bodu varu vody a bodu tuhnutí cínu:
\item[a] platinový odporový teploměr (určete konstanty $R_0, A, B$).
\item[b] termočlánek měď-konstantan (určete konstanty $a, b, c$)

\item Registrujte časový průběh termoelektrického napětí termočlánku $\epsilon(\tau)$ a odporu platinového teploměru $R(\tau)$ při ohřevu a varu vody a při tuhnutí cínu. Změřené průběhy graficky znázorněte.

\item Nakreslete graf teplotní závislosti odporu $R$ (kalibrační křivka odporového teploměru) a graf teplotní závislosti termoelektrického napětí $\epsilon$ (kalibrační křivka termočlánku).

\item Ze závislostí $\epsilon(\tau)$ a $R(\tau)$ dle bodu 2 a kalibračních hodnot dle bodu 1 určete časové závislosti $t_R(T)$ a $t_\epsilon(T)$ teplot měřených odporovým teploměrem a termočlánkem při ohřevu vody a tuhnutí cínu. Určené závislosti porovnejte.

\end{enumerate}

% ----------------------------------------------------------------------
%  Teoretická část
% ----------------------------------------------------------------------
\section{Teoretická část}
Fázové přechody jsou děje, při kterých látka přechází z jednoho skupenství do jiného. U chemicky čistých látek pozorujeme během fázového přechodu oblast s konstantní teplotou v časové závislosti, kde dochází ke změně skupenství (fáze). U ostatních látek se tato teplota o určitou hodnotu mění.

Termočlánek je složen ze dvou různých kovových vodičů. Tyto vodiče jsou na konci svařené. Podle složení těchto vodičů provádíme kalibraci. Aparatura termočlánku je zobrazena na obrázku.




% ----------------------------------------------------------------------
%  Výsledky a zpracování měření
% ----------------------------------------------------------------------
\section{Výsledky a zpracování měření}

\subsection{Laboratorní podmínky}

    Měření bylo prováděno za laboratorních podmínek uvedených v tabulce \ref{tab:lab_pod}. Pro naše měření je ale důležité, aby hodnoty (jako například teplota vody procházející  trubicí) byly co možná nejvíce podobné po celou dobu.

    \begin{table}[h]
        \centering
        \begin{tabular}{|c|c|c|} 
        \hline
            t / °C & p / hPa & vlhkost / \%RH  \\ 
        \hline
            22,9(4)   & 977,2(20)   & 34,7(25)            \\
        \hline
        \end{tabular}
        \caption{Laboratorní podmínky}
        \label{tab:lab_pod}
    \end{table}

\subsection{Podnadpis}

    
% ----------------------------------------------------------------------
%  Diskuse výsledků
% ----------------------------------------------------------------------			
\section{Diskuse výsledků}

% ----------------------------------------------------------------------
%  Závěr
% ----------------------------------------------------------------------
\section{Závěr}
