% ----------------------------------------------------------------------
%  Pracovní úkoly
% ----------------------------------------------------------------------
\section{Pracovní úkoly}

\begin{enumerate}
\item Okalibrujte pomocí bodu tání ledu, bodu varu vody a bodu tuhnutí cínu:
\item[a] platinový odporový teploměr (určete konstanty $R_0, A, B$).
\item[b] termočlánek měď-konstantan (určete konstanty $a, b, c$)

\item Registrujte časový průběh termoelektrického napětí termočlánku $\epsilon(\tau)$ a odporu platinového teploměru $R(\tau)$ při ohřevu a varu vody a při tuhnutí cínu. Změřené průběhy graficky znázorněte.

\item Nakreslete graf teplotní závislosti odporu $R$ (kalibrační křivka odporového teploměru) a graf teplotní závislosti termoelektrického napětí $\epsilon$ (kalibrační křivka termočlánku).

\item Ze závislostí $\epsilon(\tau)$ a $R(\tau)$ dle bodu 2 a kalibračních hodnot dle bodu 1 určete časové závislosti $t_R(T)$ a $t_\epsilon(T)$ teplot měřených odporovým teploměrem a termočlánkem při ohřevu vody a tuhnutí cínu. Určené závislosti porovnejte.

\end{enumerate}

% ----------------------------------------------------------------------
%  Teoretická část
% ----------------------------------------------------------------------
\section{Teoretická část}
Fázové přechody jsou děje, při kterých látka přechází z jednoho skupenství do jiného. U chemicky čistých látek pozorujeme během fázového přechodu oblast s konstantní teplotou v časové závislosti, kde dochází ke změně skupenství (fáze). U ostatních látek se tato teplota o určitou hodnotu mění.

\textbf{Termočlánek} je složen ze dvou různých kovových vodičů. Tyto vodiče jsou na konci svařené. Podle složení těchto vodičů provádíme kalibraci. Aparatura termočlánku je zobrazena na obrázku \ref{fig:aparatura-termoclanku} podle [1]. Oblast $S_1$ je srovnávací známé teploty $t_1$. Teplota $t_2$ je měřená teplota. Výhodou je malá tepelná kapacita a tedy nízké ovlivnění teploty měřeného vzorku.

\begin{figure}[h]
    \centering
    \includegraphics[width=0.5\linewidth]{8 - Kalibrace odporového teploměru a termočlánku//Prototkol - kalibrace teploměru//img/Aparatura termočlánku.png}
    \caption{Aparatura termočlánku}
    \label{fig:aparatura-termoclanku}
\end{figure}

\textbf{Odporový teploměr} měří na základě změny odporu kovu s teplotou. Standardní používané teploměry se vyrábějí z platiny.

V této práci určujeme konstanty $R_0, A, B, a, b, c$ v rovnicích, které popisují závislosti těchto dějů. Pro termočlánek platí v nejjednodušším případě aproximativní kvadratická teplotní závislost

\begin{equation}
    \epsilon = a + b(t_2 - t_1) + c(t_2 - t_1)^2
\end{equation}

kde $\epsilon$ je termoelektrické napětí a $(t_2 - t_1)$ je rozdíl teplot na obou koncích vodičů termočlánku.

Pro platinový odporový teploměr platí závislost odporu $R$ na teplotě $t$ platí

\begin{equation}
    R = R_0 (1 + At + Bt^2)
\end{equation}

Teplotu varu vody je možné vyjádřit jako

\begin{equation}
    t_p = 100 + 28,0216 \left( \frac{p}{p_0} - 1 \right) - 11,642 \left( \frac{p}{p_0} - 1 \right)^2 + 7,1 \left( \frac{p}{p_0} - 1 \right)^3
\end{equation}

kde $p_0$ je normální atmosférický tlak $p_0 = 1,01325 \cdot 10^5 Pa$ a $p$ naměřený tlak.

% ----------------------------------------------------------------------
%  Výsledky a zpracování měření
% ----------------------------------------------------------------------
\section{Výsledky a zpracování měření}

\subsection{Laboratorní podmínky}

    Měření bylo prováděno za laboratorních podmínek uvedených v tabulce \ref{tab:lab_pod}. Pro nás je, ale důležitý naměřený tlak, který ovlivňuje například teplotu varu vody.

    \begin{table}[h]
        \centering
        \begin{tabular}{|c|c|c|} 
        \hline
            t / °C & p / hPa & vlhkost / \%RH  \\ 
        \hline
            23,5(4)   & 984(2)   & 37(3)            \\
        \hline
        \end{tabular}
        \caption{Laboratorní podmínky}
        \label{tab:lab_pod}
    \end{table}

\subsection{Tání ledu}
Nejprve změříme odpor při teplotě $0 \; ^\circ C$. Umístíme směs ledu a vody do termosky a měříme vzniklé termoelektrické napětí a zaznamenáváme odpor po časových intervalech. Čekáme, než se ustálí hodnota odporu. Teplota této směsy je právě $0 ^\circ C$, proto ustálená hodnota odporu odpovídá hodnotě $R_0$ z rovnice (2). Pro odporový teploměr dostáváme $R_0 = 101,0(5) \; \Omega$. Nepřesnost měření odporu je 0,5 \% plus 0,1 \% z rozsahu. Tato závislost je zobrazena na obrázku \ref{fig:odpor-na-teplote-led}.

\begin{figure}[h]
    \centering
    \includegraphics[width=0.5\linewidth]{8 - Kalibrace odporového teploměru a termočlánku//Prototkol - kalibrace teploměru//img/Závislost R na t, tání ledu.png}
    \caption{Závislost odporu na čase pro tání ledu}
    \label{fig:odpor-na-teplote-led}
\end{figure}

Pro termočlánek při rozdílu teplot $0 \; ^\circ C$ dostáváme $\epsilon = 1,4(2) \cdot 10^{-6} \; V$. Udávaná chyba zařízení je 90 ppm z měření a 35 ppm z rozsahu. Závislost je promítnuta do grafu.

Na obrázku \ref{fig:napeti-na-cas-led} vidíme časovou závislost naměřeného napětí.

\begin{figure}[h]
    \centering
    \includegraphics[width=0.5\linewidth]{8 - Kalibrace odporového teploměru a termočlánku//Prototkol - kalibrace teploměru//img/Závislost epsilon na čase, led.png}
    \caption{Časová závislost napětí tání ledu}
    \label{fig:napeti-na-cas-led}
\end{figure}


\subsection{Var vody}

Jeden konec vodiče termočlánku necháme v termosce, která má teplotu $t_1 = 0 \; ^\circ C$. Druhý konec umístíme nad zahřívanou vodu. Hodnota varu vody je závislá na atmosférickém tlaku, kterou spočteme podle (3) dosazením laboratorního tlaku

\begin{equation}
    \nonumber
    t_p = 99,2(4) \; ^\circ C
\end{equation}

kde nejistota je odhadnuta podle metody přenosu chyb.

Pro odporový teploměr dostáváme při teplotě varu $R = 138,5(7) \; \Omega$. Tato závislost je zobrazena na grafu \ref{fig:odpor-na-teplote-var}.

\begin{figure}[h]
    \centering
    \includegraphics[width=0.5\linewidth]{8 - Kalibrace odporového teploměru a termočlánku//Prototkol - kalibrace teploměru//img/Závislost R na t, var vody.png}
    \caption{Závislost odporu na čase pro var vody}
    \label{fig:odpor-na-teplote-var}
\end{figure}

Pro termočlánek dostáváme při rozdílu teplot odpovídající teplotě varu hodnotu $\epsilon = 4,24(5) \cdot 10^{-4} \; V$.

Na obrázku \ref{fig:napeti-na-case-var} je znázorněný časové průběh naměřeného napětí. V čase 885 s bylo vypnuto topení, což se projevuje na grafu ukončením růstu.

\begin{figure}[h]
    \centering
    \includegraphics[width=0.5\linewidth]{8 - Kalibrace odporového teploměru a termočlánku//Prototkol - kalibrace teploměru//img/Závislost epsilon na čase, var.png}
    \caption{Časová závislost napětí var vody}
    \label{fig:napeti-na-case-var}
\end{figure}

\newpage
\subsection{Tavení cínu}

Bod tuhnutí cínu nastává při teplotě $t_c = 232 \; ^\circ C$. Hodnota odporu je při této teplotě rovna $R = 187,1(9) \; \Omega$. Hodnota $\epsilon = 1,095(1) \cdot 10^{-2} \; V$. Závislost odporu na čase je v grafu \ref{fig:odpor-na-teplote-cin}.

\begin{figure}[h]
    \centering
    \includegraphics[width=0.5\linewidth]{8 - Kalibrace odporového teploměru a termočlánku//Prototkol - kalibrace teploměru//img/Závislost R na t, tavení cínu.png}
    \caption{Závislost odporu na čase pro tavení cínu}
    \label{fig:odpor-na-teplote-cin}
\end{figure}

V grafu \ref{fig:napeti-na-case-cin} je zobrazena časové závislost napětí během tavení cínu. V čase 2760 s bylo zapnuto zahřívání, což je vidět na grafu.

\begin{figure}[h]
    \centering
    \includegraphics[width=0.5\linewidth]{8 - Kalibrace odporového teploměru a termočlánku//Prototkol - kalibrace teploměru//img/Závislost epsilon na čase, cín.png}
    \caption{Časová závislost napětí tavení cínu}
    \label{fig:napeti-na-case-cin}
\end{figure}

\newpage
\subsection{Kalibrační křivky}
Z naměřených hodnot odporu na napětí můžeme sestavit kalibrační křivku. Hodnoty jsou zaneseny do grafu \ref{fig:krivka-odpor} odporového teploměru a \ref{fig:krivka-termoclanek} termočlánku a je zde také sestrojen polynomiální fit se stupněm 2. Odtud získáme hledané konstanty.

\begin{figure}[h]
    \centering
    \includegraphics[width=0.8\linewidth]{8 - Kalibrace odporového teploměru a termočlánku//Prototkol - kalibrace teploměru//img/Závislost R na t, parabola.png}
    \caption{Kalibrační křivka pro odporový teploměr}
    \label{fig:krivka-odpor}
\end{figure}

\begin{figure}[h]
    \centering
    \includegraphics[width=0.8\linewidth]{8 - Kalibrace odporového teploměru a termočlánku//Prototkol - kalibrace teploměru//img/Závislost epsilon na delta T, parabola.png}
    \caption{Kalibrační křivka termočlánku}
    \label{fig:krivka-termoclanek}
\end{figure}

Po dosazení do rovnice (1) tedy máme

\begin{align*}
    a = 1,4(2) \cdot 10^{-6} \; V\\
    b = -2,8(1) \cdot 10^{-5} \; \frac{V}{^\circ C}\\
    c = 3,2(2) \cdot 10^{-7} \; \frac{V}{^\circ C^2} \\
\end{align*}

kde chyba koeficientů b, c je 5 \%.

Po dosazení do rovnice (2) dostáváme

\begin{align*}
    R_0 = 101,0(5) \; \Omega \\
    A = 0,38(2) \; \frac{1}{^\circ C} \\
    B = -2,9(1) \cdot 10^{-5} \; \frac{1}{^\circ C^2} \\
\end{align*}

kde chyba koeficientů A, B je opět 5 \%.
\newpage
\subsection{Časové závislosti}

S využitím časových závislostí odporu a napětí a rovnic s vypočítanými koeficienty, můžeme sestrojit graf závislosti teploty na čase \ref{fig:teplota-na-case-var} a \ref{fig:telota-na-case-cin}.

\begin{figure}[h]
    \centering
    \includegraphics[width=0.5\linewidth]{8 - Kalibrace odporového teploměru a termočlánku//Prototkol - kalibrace teploměru//img/Závislost teploty na čase, var.png}
    \caption{Časová závislost teploty pro var vody}
    \label{fig:teplota-na-case-var}
\end{figure}

\begin{figure}[h]
    \centering
    \includegraphics[width=0.5\linewidth]{8 - Kalibrace odporového teploměru a termočlánku//Prototkol - kalibrace teploměru//img/Závislost teploty na čase, cín.png}
    \caption{Časová závislost teploty pro var vody}
    \label{fig:telota-na-case-cin}
\end{figure}

\newpage
% ----------------------------------------------------------------------
%  Diskuse výsledků
% ----------------------------------------------------------------------			
\section{Diskuse výsledků}

Z grafů časových závislostí je vidět výrazná podobnost při měření odporu a napětí. Při porovnání časových závislostí teplot pro var vody se vypočtené křivky příliš neshodují. Zřejmě není dobře dosazeno do rovnice pro výpočet teploty z odporu. Na druhou stranu pro tavení cínu zde pozorujeme korelaci obou křivek. Drobné odchylky mohou být způsobeny nepřesnostmi popsané dále.

U vypočítaných koeficientů zde byla určena také nepřesnost daná nepřesností měřících přístrojů. Konkrétní hodnoty jsou popsány výše. Vliv na přesnost výsledků mají však i jiné faktory, než jenom tento. Je třeba uvážit, že teploměry mohou mít navzájem zpoždění, tedy nemusejí ve stejnou chvíli zobrazovat aktuální hodnoty. Navíc samotné teploměry pracují s určitou nepřesností, která není uvažována. Dále chyba může vznikat způsobem tohoto zapojení. Bylo by možné použít více bodový teploměr, který by měřený výsledek zpřesňoval díky více počtů měřidel. Samozřejmě uvažujeme dokonale čistou aparaturu i vzorky, ve skutečnosti však tyto vzorky již mohou být kontaminované nečistoty. Nepoužíváme míchadlo a teplotu měříme v jednom bodě, což může například u destilované vody měřit jednu teploty, zatímco v kádince může být u dna jiná teplota, než u hladiny. Nakonec můžeme také uvážit lidský faktor, díky kterému se opisované hodnoty mohou nepatrně lišit.

U grafů je možné vidět nepořádek v označeních veličin způsobené jedním typem používaném v zadání a druhým obecně používaných. Proto je dobré podívat se i na jednotku a předejít tak nedorozuměním. Například u symbolu $T$ je jednotka $^\circ C$, kde bychom zřejmě mohly uvažovat $K$, ale protože se jedná o rozdíl teplot, jednotky nehrají roli.

\newpage
% ----------------------------------------------------------------------
%  Závěr
% ----------------------------------------------------------------------
\section{Závěr}

V této práci jsme okalibrovali pomocí bodu tání ledu, bodu varu vody a bodu tuhnutí cínu odporový teploměr a termočlánek. získali jsme tyto koeficienty

\begin{align*}
    a = 1,4(2) \cdot 10^{-6} \; V\\
    b = -2,8(1) \cdot 10^{-5} \; \frac{V}{^\circ C}\\
    c = 3,2(2) \cdot 10^{-7} \; \frac{V}{^\circ C^2} \\
\end{align*}

\begin{align*}
    R_0 = 101,0(5) \; \Omega \\
    A = 0,38(2) \; \frac{1}{^\circ C} \\
    B = -2,9(1) \cdot 10^{-5} \; \frac{1}{^\circ C^2} \\
\end{align*}

Dále jsme zaregistrovali časový průběh termoelektrického napětí termočlánku $\epsilon (\tau)$ a odporu platinového teploměru $R(\tau)$ při ohřevu a varu vody a při tuhnutí cínu a graficky znázornili.

Nakreslili jsme graf teplotní závislosti odporu $R$ a graf teplotní závislosti termoelektrického napětí $\epsilon$.

 Nakonec jsme určili časové závislosti $t_R(\tau)$ a $t_\epsilon(\tau)$ teplot měřených odporovým teploměrem a termočlánkem při ohřevu vody a tuhnutí cínu.