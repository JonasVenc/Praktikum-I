% ----------------------------------------------------------------------
%  Pracovní úkoly
% ----------------------------------------------------------------------
\section{Pracovní úkoly}

\begin{enumerate}
\item Určete rychlost šíření podélných zvukových vln v mosazné tyči metodou Kundtovy trubice. Z naměřené rychlosti zvuku stanovte modul pružnosti v tahu E materiálu tyče.

\item Změřte rychlost zvuku ve vzduchu a v oxidu uhličitém pomocí uzavřeného resonátoru. Výsledky měření zpracujte metodou lineární regrese a graficky znázorněte.

\item Vypočítejte Poissonovu konstantu oxidu uhličitého z naměřené rychlosti zvuku.

\end{enumerate}
% ----------------------------------------------------------------------
%  Teoretická část
% ----------------------------------------------------------------------
\section{Teoretická část}

% ----------------------------------------------------------------------
%  Výsledky a zpracování měření
% ----------------------------------------------------------------------
\section{Výsledky a zpracování měření}

\subsection{Laboratorní podmínky}

    Měření bylo prováděno za laboratorních podmínek uvedených v tabulce \ref{tab:lab_pod}. Pro naše měření je ale důležité, aby hodnoty (jako například teplota vody procházející  trubicí) byly co možná nejvíce podobné po celou dobu.

    \begin{table}[h]
        \centering
        \caption{Laboratorní podmínky}
        \label{tab:lab_pod}
        \begin{tabular}{|c|c|c|} 
        \hline
            t / °C & p / hPa & vlhkost / \%RH  \\ 
        \hline
            22,9(40)   & 977,2(20)   & 34,7(25)            \\
        \hline
        \end{tabular}
    \end{table}

\subsection{Podnadpis}

    
% ----------------------------------------------------------------------
%  Diskuse výsledků
% ----------------------------------------------------------------------			
\section{Diskuse výsledků}

% ----------------------------------------------------------------------
%  Závěr
% ----------------------------------------------------------------------
\section{Závěr}
